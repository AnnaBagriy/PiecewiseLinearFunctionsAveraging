% !TEX TS-program = pdflatex
% !TEX encoding = UTF-8 Unicode
\documentclass[12pt]{article}

\usepackage[T1]{fontenc}
\usepackage[utf8]{inputenc}
\usepackage[english,ukrainian]{babel}

\usepackage[pdftex]{graphicx}
\usepackage{pdfpages}
\usepackage[a4paper,margin=2cm]{geometry}

%% Useful packages
\usepackage{amsmath}
\usepackage{graphicx}
\usepackage{hyperref}
\usepackage{systeme}
\usepackage{titletoc}
\usepackage{array}
\usepackage{subfig}
\usepackage[font=small,labelfont=bf]{caption}
\usepackage{pst-node}
%\usepackage{auto-pst-pdf}
\usepackage{tikz-cd} 
\usepackage{amsfonts} 
\usepackage{listings}
\usepackage{amsthm}
\usepackage{etoolbox}

\usepackage{minted}    %% package for inserting code
\usemintedstyle{perldoc} %{manni}
\setminted{fontsize=\footnotesize}

\newtheorem{definition}{Означення}
\newtheorem{statement}{Твердження}
\newtheorem{theorem}{Теорема}
\newtheorem{lema}{Лема}
\newtheorem{collocation}{Наслідок}

\setlength\abovecaptionskip{-5pt}

%\numberwithin{theorem}{section}
%\numberwithin{collocation}{theorem}

%\titleformat*{\section}{\huge\bfseries}
%\titleformat*{\subsection}{\large\bfseries}

\makeindex
\begin{document}
	\begin{titlepage}
		\begin{center}
			{\largeКИЇВСЬКИЙ НАЦІОНАЛЬНИЙ УНІВЕРСИТЕТ ІМЕНІ ТАРАСА ШЕВЧЕНКА}\\
			{\Large Механіко-математичний факультет \\
				Кафедра геометрії, топології та динамічних систем}
		\end{center}
		\leavevmode \\
		\leavevmode \\
		\leavevmode \\
		\leavevmode \\
		\begin{center}
			{\LARGE  Курсова робота\\}
			на тему: \\
			\leavevmode \\
			{\Huge \textbf{Топологічна стійкість усереднень функцій двох змінних}}			
		\end{center}
		\leavevmode \\
		\leavevmode \\
		\leavevmode \\
		\leavevmode \\	
		\leavevmode \\
		\leavevmode \\
		\leavevmode \\
		\leavevmode \\	
		\leavevmode \\
		\leavevmode \\
		\leavevmode \\
		\leavevmode \\     
		\begin{tabular}{p{7cm}p{12cm}}
			\, & {\large Студентки I курсу магістратури} \\
			\, & {\large Напряму підготовки ``Математика''} \\
			\, & {\large Багрій А. Г.} \\
			\, & \, \\
			\, & {\large Науковий керівник:} \\
			\, & {\large доктор фізико-математичних наук, } \\
			\, & {\large професор Пришляк О. О.} \\
		\end{tabular}
		\leavevmode \\
		\leavevmode \\
		\leavevmode \\
		\leavevmode \\
		\vfill
		\begin{center}
			{\Large Київ 2022}
		\end{center}
	\end{titlepage}
	
\section{Вступ}
	

\hspace{0.5cm} Лінійний фільтр – це така динамічна система, яка застосовує певний лінійний оператор до вхідного сигналу для виділення або відкидання певних частот сигналу та інших функцій по обробці вхідного сигналу. Лінійні фільтри широко застосовуються в електроніці, цифровій обробці сигналів і зображень, в оптиці, теорії управління та інших областях. 

Нехай вхідний сигнал задано кусково-лінійною фунцією $f:\mathbb{R}^2\rightarrow \mathbb{R}$ із скінченною кількістю локальних екстремумів. Лінійний фільтр визначається імпульсною перехідною функцією $h(x,y)$. Тоді дія лінійного фільтру на вхідний сигнал визначається як згортка $f(x,y)\ast h(x,y)$. Якщо $\iint_{\mathbb{R}}^{}h(t,s)dt ds=1$, то $h(x,y)$ можна розглядати як щільність певної ймовірнісної міри, а згортку $f(x,y)\ast h(x,y)$ – як усереднення по цій мірі.

При застосуванні фільтру важливим є питання збереження форми вхідного сигналу. Необхідно визначити, які умови на вхідну функцію $f$ та міру гарантують топологічну еквівалентність між функцією $f$ та її усередненням $f\ast h$. Зокрема, вимагається, щоб кількість екстремумів усереднення $f\ast h$ зберігалась.

Нехай $\mu$ певна ймовірнісна міра на $X=[0,1]\times [0,1]$. Тоді для будь-якої вимірної функції $f:\mathbb{R}^2\rightarrow \mathbb{R}$ та числа $\alpha$ визначається вимірна функція $f_{\alpha}:\mathbb{R}^2\rightarrow \mathbb{R}$:

\begin{equation} 
	\label{int_averaging}
	f_\alpha(x,y)= \int_{X}^{} f(x-u\alpha, y-v\alpha)d\mu,
\end{equation}

\hspace{-0.6cm}яка називається \emph{$\alpha$-усередненням} функції відносно міри $\mu$ із заданим параметром $\alpha$. 

Розглянемо випадок, коли $\mu$ – дискретна ймовірнісна міра на $X=[0,1]\times [0,1]$ зі скінченним носієм. Тобто $\exists t_{1},...,t_{k} \in X, t_{i}=(u_{i},v_{i}), i=\overline{1,k}$ – точки з $X$ такі, що для довільної борелівської підмножини $A\in X$: 
$$\mu(A)=\sum_{t_i \in A}^{}\mu(t_i).$$

Тоді усереднення функції $f:\mathbb{R}^2\rightarrow \mathbb{R}$ матиме вигляд 

\begin{equation} 
	\label{discrete_averaging}
	f_\alpha(x,y)=\sum_{i=1}^{k}f(x-u_{i}\alpha, y-v_{i}\alpha)\mu(t_{i}).
\end{equation}

\begin{definition}
	Дві неперервні фунції $f:\mathbb{R}^2\rightarrow \mathbb{R}$ та $g:\mathbb{R}^2\rightarrow \mathbb{R}$ називаються \textbf{топологічно еквівалентними}, якщо існують гомеоморфізми, що зберігають орієнтацію 
\end{definition}

\begin{equation} 
	\phi:\mathbb{R}^2\rightarrow \mathbb{R}^2, \quad \psi:\mathbb{R}\rightarrow \mathbb{R}
\end{equation} такі, що $\psi \circ f=g \circ \phi$, тобто наступна діаграма є комутативною

\[\begin{tikzcd}
	\mathbb{R}^2 \arrow{r}{f} \arrow[swap]{d}{\phi} & \mathbb{R} \arrow{d}{\psi} \\
	\mathbb{R}^2 \arrow{r}{g} & \mathbb{R}
\end{tikzcd}
\]

\begin{definition}
	Нехай $f:\mathbb{R}^2 \rightarrow \mathbb{R}$ – неперервна функція і $\mu$ – ймовірнісна міра на $[-1,1]\times[-1,1]$. Функція $f$ називається \textbf{топологічно стійкою} відносно усереднень по мірі $\mu$, якщо $\exists \varepsilon>0$ таке, що для будь-якого $\alpha \in [0, \varepsilon)$ функції $f$ та $f_{\alpha}$ є топологічно еквівалентними. Зокрема, якщо $\alpha=0$, то $f_\alpha=f$.
\end{definition}

% 2
\setcounter{secnumdepth}{1}
\section{Деякі результати про топологічну стійкість функцій}

Введемо деякі поняття, які необхідні для дослідження усереднень функцій.

\begin{definition}
	Назвемо \textbf{опуклим клітковим розбиттям} простору $\mathbb{R}^2$ сім`ю опуклих многокутників $C=\{c_i\}_{i \in \mathbb{N}}$, яке задовольняє наступним умовам:
	
	\begin{enumerate}
		\item $\bigcup\limits_{i}^{} c_{i}=\mathbb{R}^2$,
		\item $int(c_i) \cap int(c_j)=\emptyset, \forall c_i, c_j, i,j \in \mathbb{N}$,
		\item $c_i \cap c_j$ – спільне ребро, спільна вершина або порожня множина.
	\end{enumerate}
\end{definition}

\begin{definition}
	\textbf{k-скелет} розбиття $C$ визначається наступним чином:
	\begin{itemize}
		\item $C^0$ – вершини,
		\item $C^1$ – внутрішність ребра,
		\item $C^2$ – внутрішність граней.
	\end{itemize}
\end{definition}

Через $S \wedge Q$ позначимо опукле кліткове розбиття, що складається з усіх непорожніх попарних перетинів клітин, тобто
\begin{equation}
	S \wedge Q = \{s_{i} \cap q_{j}\}_{i,j \in \mathbb{N}}.
\end{equation}

Розглянемо розбиття $C$, що зсунуте на певний вектор $\delta_i=(u_i,v_i), i\leq k, k\in\mathbb{N}$. Позначимо його як $C+\delta_i=\{u_j+\delta_i\}_{i,j\in\mathbb{N}}$. Тоді можна визначити таке опукле кліткове розбиття, що утворене непорожніми перетинами клітин для всіх зсувів $\delta_i$: $C+\delta:=\bigwedge_{i,j} (c_j+\delta_i)$.

\begin{definition}
	Функція $f:\mathbb{R}^2\rightarrow \mathbb{R}$ називається \textbf{кусково-лінійною} відносно розбиття $C(f)$, якщо на кожній клітині $c_i \in C(f)$ задана лінійна функція $f_i$:

\begin{equation}
	f(x,y)=
	\begin{cases}
		&f_{1}(x,y), \quad (x,y) \in c_1,\\
		&f_{2}(x,y), \quad (x,y) \in c_2,\\
		&\vdots\\
		&f_{n}(x,y), \quad (x,y) \in c_n,\\
	\end{cases}\
\end{equation}
$n \in \mathbb{N}, n\geq 3, \ f_i(x,y)=f_j(x,y) \ \forall (x,y)\in c_i \cap c_j$.
	
\end{definition}


\begin{statement}
	Нехай $f:\mathbb{R}^2 \rightarrow \mathbb{R}$ – кусково-лінійна функція, $C(f)$ – її опукле кліткове розбиття. Тоді якщо $(x, y) \in R^2$ – точка локального екстремуму заданої функції і $\exists c \in C(f) \text{– відкрита клітина:}\ x \in c$, то $f(x,y)=f(\bar{c})$, тобто екстремум досягається на замиканні клітини.
\end{statement}

Нехай $\mu$ – дискретна ймовірнісна міра на $[-1,1]\times[-1,1]$ зі скінченним носієм $t_{1},...,t_{k}$, $p_{i}:=\mu (t_{i}): \sum_{i=1}p_i=1$. Усереднення \eqref{discrete_averaging} – це лінійна комбінація функцій $f_{i}(x,y)$, початок координат яких зміщений у точки $\omega^\alpha_i:=(u_{i}\alpha, v_{i}\alpha)$, із відповідними коефіцієнтами $p_{i}$. Позначимо ці функції як $f^i(x, y):=f(x-u_{i}\alpha, y-v_{i}\alpha)$. Вони визначені на розбитті $c^\alpha_i=\{c_j+w^\alpha_i\}_{j\in \mathbb{N}}$. Тоді

\begin{equation}
	f^i(x,y)=
	\begin{cases}
		&f^i_{1}(x,y), \quad (x,y) \in c_1+w^\alpha_i,\\
		&f^i_{2}(x,y), \quad (x,y) \in c_2+w^\alpha_i,\\
		&\vdots\\
		&f^i_{n}(x,y), \quad (x,y) \in c_n+w^\alpha_i,\\
	\end{cases}\
\end{equation} $i=\overline{1,k}$. Лінійна комбінація кусково-лінійних функцій – це також кусково-лінійна функція. Отже, $f_\alpha$ – кусково-лінійна функція відносно опуклого кліткового розбиття $C(f_\alpha)$.

Функція $f_\alpha$ може набувати свого мінімуму на замиканні обмежених клітин з $C^\alpha$ або на обмежених k-скелетах розбиття.

\begin{definition}
	Функція $f$ називається \textbf{однорідною}, якщо $f(tx,ty)=tf(x,y)\ \forall t\geq 0$.
\end{definition}

%Розглянемо множину усіх перетинів векторів $l^i_j, i=\overline{1,k}, j=\overline{1,n}$ і позначимо її $I$. Позначимо множину усіх точок $(u_{i}\alpha, v_{i}\alpha)$ як $O$. Розглянемо опуклу оболонку Conv$(I \cup O)$ множини точок $I \cup O$. Тобто таку мінімальну опуклу множину, що містить $I \cup O$. Якщо $\exists x_p, x_q \in I \cup O, p,q \in \mathbb{N}$ такі, що $x_p \in l^i_j$ і $x_p \in l^i_j$ для деякого $l^i_j$, то $x_p, x_q$ утворюють відрізок $[x_p, x_q]=\{ x \in \mathbb{R} \vert x_p \leq x \leq x_q \} \subset \text{Conv}(I \cup O)$. Якщо існують такі відрізки $v_1, v_2,...,v_n \subset \text{Conv}(I \cup O), n\geq 3$, утворені точками з $I \cup O$, що з`єднані попарно і утворюють замкнений шлях, то ці відрізки утворюють многокутник, який належить Conv$(I \cup O)$. Позначимо усіх замкнені підмножини, що містяться у Conv$(I \cup O)$ як C. Її елементами будуть точки, відрізки, трикутники і т.д.

\begin{theorem}
	Нехай $f_\alpha:\mathbb{R}^2\rightarrow \mathbb{R}$ – кусково-лінійна функція з розбиттям $C(f_\alpha)$, а $f_\beta:\mathbb{R}^2\rightarrow \mathbb{R}$ – кусково-лінійна функція з розбиттям $C(f_\beta)$, $\alpha, \beta>0$. Тоді $C\big(f_\alpha(x,y)\big)= C\big(\frac{\alpha}{\beta}f_\beta(\frac{\beta x}{\alpha}, \frac{\beta y}{\alpha})\big) \ \forall \alpha, \beta>0$.
\end{theorem}

\begin{proof}
	
	%\begin{equation}
	%	C(f_\alpha(x,y))=\phi_\alpha(C(f_{\alpha_0}(x,y)))=\{\phi_\alpha(c): \ c \in C(f_{\alpha_0})\}
	%\end{equation}
	
	Введемо функцію $\phi_\alpha:\mathbb{R}^2\rightarrow\mathbb{R}^2:\ \phi_\alpha(x,y)=(\alpha x,\alpha y)$. Для заданої міри $\mu$ та параметру $\alpha$ усереднення має вигляд
	
	\begin{equation} 
		f_\alpha(x,y)=\sum_{i=1}^{k}f(x-u_{i}\alpha, y-v_{i}\alpha)p_{i}.
	\end{equation}
	
	При $\alpha=1$ маємо
	
	\begin{equation} 
		f_1(x,y)=\sum_{i=1}^{k}f(x-u_{i}, y-v_{i})p_i.
	\end{equation}
	
	Для $\forall \alpha>0$ з однорідності маємо
	
	\begin{equation} 
		f_1(\alpha x,\alpha y)=\sum_{i=1}^{k}f(\alpha x-u_{i}, \alpha y-v_{i})p_i=\alpha \sum_{i=1}^{k}f(x-\frac{u_{i}}{\alpha}, y-\frac{v_{i}}{\alpha})p_i=\alpha f_{\frac{1}{\alpha}}.
	\end{equation}
	
	\begin{equation} 
		\alpha f_{\frac{1}{\alpha}}=f_1 \circ \phi_{\alpha} \implies f_{\alpha}=\alpha f_1 \circ \phi_{\frac{1}{\alpha}}.
	\end{equation}
	
%	$f_{\alpha_0} := f_1 \circ \phi_{\frac{1}{\alpha'}} \implies C(f_\alpha)=\alpha C(f_{\alpha_0})$.
	
	Тоді $f_1=\frac{1}{\alpha}f_\alpha \circ \phi_\alpha$. Аналогічно $\forall \beta>0$ $f_1=\frac{1}{\beta}f_\beta \circ \phi_\beta$.
	
	\begin{equation} 
		\frac{1}{\alpha}f_\alpha \circ \phi_\alpha=\frac{1}{\beta}f_\beta \circ \phi_\beta \implies f_\alpha=\frac{\alpha}{\beta}f_\beta \circ \phi_\beta \circ \phi_\frac{1}{\alpha}=\frac{\alpha}{\beta} f_\beta(\frac{\beta x}{\alpha},\frac{\beta y}{\alpha}).
	\end{equation}
	
	Тоді $C\big(f_\alpha(x,y)\big)=C\big(\frac{\alpha}{\beta} f_\beta(\frac{\beta x}{\alpha},\frac{\beta y}{\alpha})\big).$
	
\end{proof}

\begin{collocation}
	Нехай $f:\mathbb{R}^2\rightarrow \mathbb{R}$ – однорідна кусково-лінійна функція на розбитті $C(f)$. Якщо $\exists \alpha_0>0$ таке, що $f_{\alpha_0}$ має один екстремум, то $\forall \alpha>0, \alpha<\alpha_0$ $f_\alpha$ теж має один екстремум.
\end{collocation}

\begin{collocation}
	Якщо $f:\mathbb{R}^2\rightarrow \mathbb{R}$ – однорідна кусково-лінійна функція і $\exists \alpha_0>0$ таке, що $\exists!(x,y)\in \mathbb{R}^2$ – точка локального екстремуму $f_\alpha$, то $f$ – топологічно стійка відносно усереднення по мірі $\mu$ для $\forall \alpha\leq\alpha_0, \alpha>0$.
\end{collocation}

\begin{collocation}
	Якщо $f:\mathbb{R}^2\rightarrow \mathbb{R}$ – однорідна кусково-лінійна функція і $f$  не є топологічно стійкою відносно усереднення по мірі $\mu$, то $\forall \alpha>0 \ \exists k\in \{1,2\}:  \exists c_i\in C^k: \ f_\alpha$ приймає постійне значення на $c_i$.
%	 $\exists k\in \{0,1,2\}, \ C^k\in C^\alpha, \exists g:\mathbb{R}^2\rightarrow\mathbb{R}: g(C^k)=m$.
\end{collocation}


% 3
\section{Приклади} \label{quad}


%Для зручності демонстрації прикладів розглянемо лінії рівня функції $f$, тобто такі множини $L_c(f)=\{ (x,y): f(x,y)=c \}, c \in \mathbb{R}$.

\subsection{Приклад 1}

Розглянемо функцію

\begin{equation}
	\label{mod_function}
	f(x,y)=|x|+|y|, \quad x,y \in \mathbb{R}.
\end{equation}

Дана функція має глобальний мінімум у точці $(0,0)$. Зафіксуємо носій ймовірнісної міри у точках $t_1=(1, 1), t_2=(1, -1), t_3=(-1, -1), t_4=(-1, 1), p_{i}=1/4, i=\overline{1,4}$. Нехай $\alpha=0.5$. Тоді усереднення заданої функції матиме вигляд

\begin{equation}
	f_\alpha(x,y)=
	\begin{cases}
		&-x-y, \quad x,y \in (-\infty,-\alpha), \\
		&-x+\alpha, \quad x \in (-\infty,-\alpha), y \in [-\alpha,\alpha], \\
		&-y+\alpha, \quad x \in [-\alpha,\alpha], y \in (-\infty,-\alpha), \\
		&2\alpha, \quad x,y \in [-\alpha,\alpha], \\
		&-x+y, \quad x \in (-\infty,-\alpha), y \in (\alpha,+\infty),\\
		&y+\alpha, \quad x \in [-\alpha,\alpha], y \in (\alpha,+\infty), \\
		&x-y, \quad x \in (\alpha,+\infty), y \in (-\infty, -\alpha), \\
		&x+\alpha, \quad x \in (\alpha,+\infty), y \in [-\alpha,\alpha], \\
		&x+y, \quad x,y \in (\alpha,+\infty). \\
	\end{cases}\
\end{equation}

Дійсно, усереднення не зберігає мінімум заданої функції з точністю до гомеоморфізму – на прямокутнику $[-\alpha,\alpha] \times [-\alpha,\alpha]$ $f_\alpha$ приймає стале значення $2\alpha$.

При візуальному аналізі результатів зручно розглядати лінії рівня функцій та вектори, на яких задаються вони задаються. Тобто множини $L_c(f)=\{ (x,y): f(x,y)=c \}, c \in \mathbb{R}$. Для подальшої демонстрації результатів використовується програма, написана на Python, яка візуалізує описану вище множину С.

\begin{figure}[h]
	\centering \includegraphics[scale=0.8]{f_mod}
	\caption{Зелений графік – вектори початкової функції, сині графіки – вектори функцій $f^i$, бірюзові лінії формують опуклу обонку}
	\label{f_mod_hull}
\end{figure}

Проаналізувавши отриманий результат та рисунок \ref{f_mod_hull}, бачимо, що множина C складається із 4 точок, 4 відрізків та одного прямокутнику. Мінімум $f_\alpha$ досягається на прямокутнику.

\newpage
Тепер проаналізуємо поведінку усереднення функції неперервно змінюючи $p_{1}, p_{2}, p_{3}$ при фіксованому $p_{4}=0.25$ врахочуючи, що $p_{1}+p_{2}+p_{3}+p_{4}=1$. Зафіксуємо ітераційний крок $s=0.005$ і будемо визначати міру як $s \leq p_{1} \leq 1-p_{4}, s \leq p_{2} \leq 1-p_{4}-p_{1}, p_{3}=1-p_{4}-p_{1}-p_{2}$. Маємо наступний ймовірнісний симплекс:

\begin{figure}[!ht]
	\centering \includegraphics[scale=0.4]{f_mod_p}
	\caption{Зелені точки на симплексі відповідають такій мірі, що усереднення зберігає один мінімум. Червоні точки навпаки – такій, де усереднення має не один мінімум.}
	\label{f_mod_p}
\end{figure}


Міра $(p_{1}, p_{2}, p_{3})$ тут визначає дві прямі, що перетинаються. При чому, точка перетину відповідає мірі $p_{i}=1/4, i=\overline{1,4}$, де мінімум усереднення – це прямокутник. Відповідно, для інших точок цих прямих мінімум усереднення – це якийсь відрізкок з множини C.

\subsection{Приклад 2}

Розглянемо іншу конфігурацію. Нехай

\begin{equation}
	f(x,y)=
	\begin{cases}
		&2x+y, \quad x \geq 0, x \geq -y, \\
		&-2x+y, \quad x \leq 0, x \leq y, \\
		&-y, \quad x \geq y, x \leq -y. \\
	\end{cases}\
\end{equation}

Лінії рівня цієї функції мають наступний вигляд:

\begin{figure}[h]
	\centering \includegraphics[scale=0.5]{f_triangle_level_set}
	\caption{f(x,y)}
	\label{f_triangle_level_set}
\end{figure}

Очевидно, що функція має глобальний мінімум у точці $(0,0)$. Зафіксуємо носій ймовірнісної міри у точках $t_1=(1, 1), t_2=(1, -1), t_3=(-1, -1), p_{1}=0.3, p_{2}=0.5, p_{3}=0.2$. Нехай $\alpha=0.5$. 


\begin{figure}[h]
	\centering \includegraphics[scale=0.7]{f_triangle_hull}
	\caption{Зелений графік – вектори початкової функції, сині графіки – вектори функцій $f^i$, бірюзові лінії формують опуклу обонку}
	\label{f_triangle_hull}
\end{figure}

З рисунку \ref{f_triangle_hull} видно, що множина C cкладається з 5 точок, 6 відрізків і одного трикутника. Подальший аналіз проведемо за допомогою іншої програми, що написана на Python. 


Розглянемо графік усереднення $f_\alpha$ в $\mathbb{R}^3$. Як бачимо, усереднення не зберігає мінімум початкової функції, натомість, усереднення набуває мінімум на одному із відрізків з C.


\begin{figure}[h]
	\centering \includegraphics[scale=0.4]{f_triangle_3d_av}
	\caption{1-ий графік – це "погляд" на функцію з площини Oyz, 1-ий графік – це "погляд" на функцію з площини Oyx }
	\label{f_triangle_3d_av}
\end{figure}

\newpage
\subsection{Приклад 3}

Розглянемо таку функцію

\begin{equation}
	f(x,y)=
	\begin{cases}
		&\frac{3x}{2}+y, \quad x \geq 0, x \geq -2y, \\
		&\frac{x}{3}-\frac{4y}{3}, \quad x \leq 0, x \leq y, \\
		&-2x+y, \quad x \geq y, x \leq -2y. \\
	\end{cases}\
\end{equation}

Лінії рівня цієї функції мають наступний вигляд:

\begin{figure}[h]
	\centering \includegraphics[scale=0.4]{f_non_sym_triangle_level_set}
	\caption{f(x,y)}
	\label{f_non_sym_triangle_level_set}
\end{figure}

Зафіксуємо носій міри аналогічно до попереднього прикладу у точках $t_1=(1, 1), t_2=(1, -1), t_3=(-1, -1)$. Проте тепер проаналізуємо поведінку усереднення неперервно змінюючи $p_{1}, p_{2}, p_{3}$ врахочуючи, що $p_{1}+p_{2}+p_{3}=1$. Зафіксуємо ітераційний крок $s=0.005$ і будемо визначати міру як $s \leq p_{1} \leq 1, s \leq p_{2} \leq 1-p_{1}, p_{3}=1-p_{1}-p_{2}$. Тоді цікавим є наступний результат. Для даної функції в заданих точках міри усереднення буде завжди зберігати мінімум. Це можна побачити подивившись на наступний ймовірнісний симплекс:

\begin{figure}[h]
	\centering \includegraphics[scale=0.3]{f_triangle_p}
	\caption{Ймовірнісний симплекс}
	\label{f_triangle_p}
\end{figure}

Такі результати показують, що існують певні достатні умови для функції та заданої міри, які гарантують топологічну стійкість.


\newpage
	% 4
	\section{Висновки}	
	
В даній роботі була сформульована задача усереднення функцій двох змінних відносно дискретної ймовірносної міри із заданим параметром. Була проаналізована поведінка усереднень для різних функцій за допомогою теоретичного підходу, а також із допомогою написаних комп`ютерних програм на мові Python. Отримані загальні результати поведінки усереднень, які надалі допоможуть вивести широки достатні умови для топологічної стійкості функцій.
\\
\\
Хочу висловити подяку члену-кореспонденту НАН України, доктору фізико-математичних наук, професору Максименко С. І. за запропоновану задачу, а також за постійне наставництво та допомогу при роботі з даною темою.
	

	%5
	\begin{thebibliography}{99}
		\addcontentsline{toc}{chapter}{Література}
		
		\bibitem{main} 
		Maksymenko, S. I.; Marunkevich, O. V. Topological stability of averagings of functions. Mat. Zh. 68 (2016), no. 5, 625-633 Ukrainian Math. J. 68 (2016), no. 5, 707-717 pp.
		
		\bibitem{measure}
		Kolmogorov, A. N.; Fomin, S. V. Элементы теории функций и функционального анализа. Fourth edition, revised. Izdat. ``Nauka'', Moscow, 1976. 543 pp.
		

	\end{thebibliography}

\newpage
\section*{Додаток}

\subsection{Програма 1}
	
\begin{minted}{python}
	
	import numpy as np
	import matplotlib.pyplot as plt
	import sys
	import import_ipynb
	
	from matplotlib import cm
	from matplotlib.ticker import LinearLocator
	
	np.set_printoptions(threshold=sys.maxsize)
	
	figure_size = 10
	plt.rcParams["figure.figsize"] = (figure_size, figure_size)
	
	x_interval = [-1, 1]
	y_interval = [-1, 1]
	
	step_interval = 100
	
	alpha = 0.5
	epsilon = 10e-10
	
	# Functions
	
	'''
	t - vector of points
	t_k = (u_k, v_k)
	
	p - vector of coefficients
	sum(p1,...,pn) = 1
	
	alpha > 0
	'''
	def f_a(x, y, t, p, alpha):
		f_a = np.zeros_like(x)
		
		for i in range(0, len(t)):
			f_a += f(x[:] - t[i, 0] * alpha, 
				y[:] - t[i, 1] * alpha) * p[i]
		
		return f_a
	
	'''
	Calculates the value of the function
	and its averaging
	
	Returns two vectors
	'''
	def calc_f(t, p, alpha):
		x = np.linspace(x_interval[0], x_interval[1], step_interval)
		y = np.linspace(y_interval[0], y_interval[1], step_interval)
		z = []
		
		X, Y = np.meshgrid(x, y)
		
		zs = np.array(f(np.ravel(X), np.ravel(Y)))
		Z = zs.reshape(X.shape)
		
		zs_a = np.array(f_a(np.ravel(X), np.ravel(Y), t, p, alpha))
		Z_a = zs_a.reshape(X.shape)
		
		return Z, Z_a
		
	'''
	Function for finding and comparing
	minimum value of the input function
	and its averaging
	
	Returns bool value whish indicates
	whether the averaging saves the extrema
	'''
	def find_min(Z, Z_alpha, print_result = False):
		elements, count = np.unique(np.floor(Z_alpha/epsilon)
		.astype(int), return_counts = True)
		duplicates = elements[count > 1]
		
		Z_min = Z.min()
		Z_a_min = Z_alpha.min()
		
		duplicate_min = np.any(duplicates == np.floor(
		Z_a_min/epsilon).astype(int))
		
		if print_result:
		print('-------')
		
		print('Z_min =', Z_min)
		print('Z_a_min =', Z_a_min)
		
		print('\nPreserves min =', not duplicate_min)
		
		print()
		
		return not duplicate_min
	
	'''
	Function for plotting the surface of the function
	
	t - vector of points
	t_k = (u_k, v_k)
	
	p - vector of coefficients
	sum(p1,...,pn) = 1
	
	alpha > 0
	
	is_averaging - indicator if we should plot the input function of its averaging
	
	UTILITIES:
	show_from_two_angles - bool variable that indicates if to show plot
	from the Ox & Oy 
	show_colorbar - bool variable that indicates if to show colorbar
	xy_angle - observation angle of XY plane
	z_angle - observation angle of Z ax
	title - title of the plot
	'''
	def plot_surface(t, p, alpha, is_averaging = False, 
		show_from_two_angles = False,
		xy_angle = 0, z_angle = 0, title = ''):
		
		show_colorbar = True
		
		if show_from_two_angles:
			fig, ax = plt.subplots(nrows=1, ncols=2, 
			figsize=(20,20), subplot_kw={"projection": "3d"})
		else:
			fig, ax = plt.subplots(
			subplot_kw={"projection": "3d"})
		
		x = np.linspace(x_interval[0], x_interval[1], step_interval)
		y = np.linspace(y_interval[0], y_interval[1], step_interval)
		z = []
		
		X, Y = np.meshgrid(x, y)
		
		if is_averaging:
			zs = np.array(f_a(np.ravel(X), np.ravel(Y), 
				t, p, alpha))
		else:
			zs = np.array(f(np.ravel(X), np.ravel(Y)))
		
		Z = zs.reshape(X.shape)
		
		if show_from_two_angles:
			for i in range(0, 2):
				angle_to_rotate = 90
		
		if i == 1:
			z_angle += angle_to_rotate
		
			ax[i].view_init(xy_angle, z_angle)
			
			surf = ax[i].plot_surface(X, Y, Z, cmap=cm.coolwarm,
			linewidth=0, antialiased=False)
			
			ax[i].zaxis.set_major_locator(LinearLocator(10))
			ax[i].zaxis.set_major_formatter('{x:.02f}')
			
			ax[i].set_xlabel('x')
			ax[i].set_ylabel('y')
		else:
			ax.view_init(xy_angle, z_angle)
		
			surf = ax.plot_surface(X, Y, Z, cmap=cm.coolwarm,
			linewidth=0, antialiased=False)
			
			ax.zaxis.set_major_locator(LinearLocator(10))
			ax.zaxis.set_major_formatter('{x:.02f}')
			
			ax.set_xlabel('x')
			ax.set_ylabel('y')
		
		if show_colorbar:
			fig.colorbar(surf, shrink=0.2, aspect=10)
		
		plt.title(title)
		plt.show()
		
		return Z
		
	'''
	Function for finding and plotting the measure
	of 4 points (with 1 fixed and 3 continuously changing)
	where averaging of the function preserves extrema and where not
	
	t - vector of points
	t_k = (u_k, v_k)
	
	alpha > 0
	step > 0
	
	UTILITIES:
	xy_angle - observation angle of XY plane
	z_angle - observation angle of Z ax
	
	
	Returns two distinct vectors:
	with measure that preserves extrema and where not
	'''
	def test_4_measures(t, alpha, step = 0.05, 
		xy_angle = 0, z_angle = 0):
		p1 = p2 = p3 = 0.0
		p4 = 0.25
		
		p_dif = 1 - p4
		
		number_of_dec = str(step)[::-1].find('.')
		
		P = np.empty(shape=(0,4), dtype=float)
		
		P_good = np.empty(shape=(0,4), dtype=float)
		P_bad = np.empty(shape=(0,4), dtype=float)
		
		for p1 in np.arange(step, p_dif, step):
			for p2 in np.arange(step, p_dif - p1, step):
			
				p3 = p_dif - p2 - p1
				
				p1 = float(f'{p1:.{number_of_dec}f}')
				p2 = float(f'{p2:.{number_of_dec}f}')
				p3 = float(f'{p3:.{number_of_dec}f}')
				
				if p3 < step or p2 < step:
					continue
				
				p = [p1, p2, p3, p4]
				
				if any((P[:] == p).all(1)):
					continue
				
				P = np.append(P, [p], axis=0)
				
				Z_f, Z_a = calc_f(t, p, alpha)
				preserves = find_min(Z_f, Z_a)
				
				if preserves:
					P_good = np.append(P_good, [p], 
					axis=0)
				else:
					P_bad = np.append(P_bad, [p], 
					axis=0)
				
				if preserves:
					P_good = np.append(P_good, [p], 
					axis=0)
				else:
					P_bad = np.append(P_bad, [p], 
					axis=0)
		
		col = ['green' if any((P_good[:] == P[i,:]).all(1)) 
			else 'red' for i in range(0, len(P))]
		
		fig = plt.figure()
		
		ax = fig.add_subplot(111, projection='3d')
		ax.view_init(xy_angle, z_angle)
		
		ax.scatter(P[:,0], P[:,1], P[:,2], c=col, label='sd')
		
		ax.set_xlabel('p1')
		ax.set_ylabel('p2')
		ax.set_zlabel('p3')
		
		#print(P)
		return P_good, P_bad
	
	'''
	Function for finding and plotting the measure
	of 3 points that are continuously changing
	where averaging of the function preserves extrema and where not
	
	t - vector of points
	t_k = (u_k, v_k)
	
	alpha > 0
	step > 0
	
	UTILITIES:
	xy_angle - observation angle of XY plane
	z_angle - observation angle of Z ax
	
	
	Returns two distinct vectors:
	with measure that preserves extrema and where not
	'''
	def test_3_measures(t, alpha, step = 0.05, 
		xy_angle = 0, z_angle = 0):
		p1 = p2 = p3 = step
		
		p_dif = 1
		
		number_of_dec = str(step)[::-1].find('.')
		
		P = np.empty(shape=(0,3), dtype=float)
		
		P_good = np.empty(shape=(0,3), dtype=float)
		P_bad = np.empty(shape=(0,3), dtype=float)
		
		for p1 in np.arange(step, p_dif, step): 
			for p2 in np.arange(step, p_dif - p1, step):
			
				p3 = p_dif - p2 - p1
				
				p1 = float(f'{p1:.{number_of_dec}f}')
				p2 = float(f'{p2:.{number_of_dec}f}')
				p3 = float(f'{p3:.{number_of_dec}f}')
				
				if p3 < step or p2 < step:
					continue
				
				p = [p1, p2, p3]
				
				if any((P[:] == p).all(1)):
					continue
				
				P = np.append(P, [p], axis=0)
				
				Z_f, Z_a = calc_f(t, p, alpha)
				preserves = find_min(Z_f, Z_a)
				
				if preserves:
					P_good = np.append(P_good, [p], 
						axis=0)
				else:
					P_bad = np.append(P_bad, [p], 
						axis=0)
				
		col = ['green' if any((P_good[:] == P[i,:]).all(1))
	 		else 'red' for i in range(0, len(P))]
		
		fig = plt.figure()
		
		ax = fig.add_subplot(111, projection='3d')
		ax.view_init(xy_angle, z_angle)
		
		ax.scatter(P[:,0], P[:,1], P[:,2], c=col)
		
		ax.set_xlabel('p1')
		ax.set_ylabel('p2')
		ax.set_zlabel('p3')
		
		return P_good, P_bad
	
	'''
	Function for demonstrating the result of averaging
	analytically and visually
	'''
	def find_and_plot_averaging(t, p, alpha, show_from_two_angles = True, 
		xy_angle = 0, z_angle = 0):
		for i in range(0, len(p)):
			p_set = p[i]
			
			Z, Z_a = calc_f(t, p_set, alpha)
			preserves = find_min(Z, Z_a, True)
			
			print('p =', p_set)
			
			_ = plot_surface(t, p_set, alpha, True, 
				show_from_two_angles, xy_angle, z_angle)
	
\end{minted}

\subsection{Програма 2}

\begin{minted}{python}
	
	import numpy as np
	import matplotlib.pyplot as plt
	import sys
	import math
	
	from scipy.spatial import ConvexHull
	
	np.set_printoptions(threshold=sys.maxsize)
	
	figure_size = 5
	plt.rcParams["figure.figsize"] = (figure_size, figure_size)
	
	x_interval = [-1, 1]
	y_interval = [-1, 1]
	
	step_interval = 100
	
	alpha = 0.5
	epsilon = 10e-5
	
	'''
	t - array of measure points
	alpha - given constant
	
	Returns an array of averaged zeroes of the function
	'''
	def average_zero(t, alpha):    
		result = np.empty(shape=(0,2), dtype=float)
		
		for i in range(0, len(t)):
		
			zero = np.empty(shape=(2), dtype=float)
			
			zero[0] = t[i, 0] * alpha
			zero[1] = t[i, 1] * alpha
			
			result = np.append(result, [zero], axis=0)
		
		return result
	
	'''
	vectors - an array of vectors
	t - array of measure points
	alpha - given constant
	
	Returns an array of averaged vectors on which
	the function if given
	'''
	def average_vectors(vectors, t, alpha):
		vectors_count = len(vectors)
		
		result = np.empty(shape=(0,vectors_count,2), dtype=float)
		
		for i in range(0, len(t)):
		
			averaged_vectors = np.empty(shape=(0,2), 
				dtype=float)
			
			for j in range(0, vectors_count):
				averaged_vector = np.empty(shape=(2), 
					dtype=float)
				
				averaged_vector[0] = vectors[j, 0]
					+ t[i, 0] * alpha
				averaged_vector[1] = vectors[j, 1] 
					+ t[i, 1] * alpha
				
				averaged_vectors = np.append(
					averaged_vectors, 
					[averaged_vector], axis=0)
			
			result = np.append(result, [averaged_vectors], 
				axis=0)
		
		return result
		
	'''
	Returns a list of rays intersections if any
	
	line1 - array with a start and an end of the line
	that respresents the first ray
	line2 - array with a start and an end of the line
	that respresents the second ray
	'''
	def find_intersection_of_rays(line1, line2):
	
		if np.array_equal(line1[0], line2[1]):
			return line1[0]
		
		elif np.array_equal(line2[0], line1[1]):
			return line2[0]
		
		result = np.empty(shape=(2), dtype=float)
		
		v1 = [line1[1][0] - line1[0][0], line1[1][1] - line1[0][1]]
		v2 = [line2[1][0] - line2[0][0], line2[1][1] - line2[0][1]]
		
		l1 = np.sqrt(v1[1]**2 + v1[0]**2)
		l2 = np.sqrt(v2[1]**2 + v2[0]**2)
		
		n1 = [v1[0] / l1, v1[1] / l1]
		n2 = [v2[0] / l2, v2[1] / l2]
		
		dx = line2[0][0] - line1[0][0]
		dy = line2[0][1] - line1[0][1]
		
		det = n2[0] * n1[1] - n1[0] * n2[1]
		
		if det != 0:
			u = (dy * n2[0] - dx * n2[1]) / det
			v = (dy * n1[0] - dx * n1[1]) / det
		
			if u >= 0 and v >= 0:
				if n1[0] == 0 or n2[0] == 0:
					m0 = n1[1] / (n1[0] + epsilon)
					m1 = n2[1] / (n2[0] + epsilon)
				else:
					m0 = n1[1] / n1[0]
					m1 = n2[1] / n2[0]
			
			b0 = line1[0][1] - m0 * line1[0][0]
			b1 = line2[0][1] - m1 * line2[0][0]
			
			x = (b1 - b0) / (m0 - m1)
			y = m0 * x + b0
			
			result = [x, y]
			
			return result
			print(result)
		
	# Main
	
	initial_vectors = np.array([[0, 1], [1, -1], [-1, -1]])
	t = np.array([[1, 0.5], [-1, 1], [0, -1]])
	
	averaged_vectors = average_vectors(initial_vectors, t, alpha)
	averaged_zero = average_zero(t, alpha)
	
	points = np.empty(shape=(0,2), dtype=float)
	
	for l in range(0, len(averaged_zero)):
		start1 = averaged_zero[l]
		end1_list = averaged_vectors[l]
		
		for i in range(0, len(initial_vectors)):
			end1 = end1_list[i]
			
			for j in range(l + 1, len(t)):
				start2 = averaged_zero[j]
				end2_list = averaged_vectors[j]
				
				for k in range(0, len(initial_vectors)):
					end2 = end2_list[k]
					
					point = 
					find_intersection_of_rays(
					[start1, end1], 
					[start2, end2])
					
					if point is not None and not 
					np.isnan(point).any():
						points = np.append(
						points, 
						[point], 
						axis=0)
	
	convexFigure = np.concatenate((points, averaged_zero))
	
	hull = ConvexHull(convexFigure)
	
	fig, (ax1, ax2) = plt.subplots(ncols=2, figsize=(10, 3))
	
	for i in range(0, len(initial_vectors)):
		x_values = [0, initial_vectors[i, 0]]
		y_values = [0, initial_vectors[i, 1]]
	
	plt.plot(x_values, y_values, color='green')
	
	x_values = [averaged_zero[0, 0], averaged_vectors[0, 2, 0]]
	y_values = [averaged_zero[0, 1], averaged_vectors[0, 2, 1]]
	
	for i in range(0, len(averaged_vectors)):
		for j in range(0, len(averaged_vectors[i])):
			x_values = [averaged_zero[i, 0], 
				averaged_vectors[i, j, 0]]
			y_values = [averaged_zero[i, 1], 
				averaged_vectors[i, j, 1]]
		
			plt.plot(x_values, y_values, color='blue')
	
	for ax in (ax1, ax2):
		ax.plot(convexFigure[:, 0], convexFigure[:, 1],
			 '.', color='k')
		
		if ax == ax1:
			ax.set_title('Given points')
		else:
			ax.set_title('Convex hull')
			for simplex in hull.simplices:
				ax.plot(convexFigure[simplex, 0], 
				convexFigure[simplex, 1], 'c')
			
			ax.plot(convexFigure[hull.vertices, 0], 
			convexFigure[hull.vertices, 1], 'o', mec='r', 
			color='none', lw=1, markersize=10)
	
	plt.show()
	
	
\end{minted}
	
\end{document}